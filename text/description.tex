
Benchmark-based evaluation is often used to predict the effect of a set of code transformations on the performance of actual applications that resemble the benchmark used in the evaluation. An issue with many of the performance evaluations of \FDO-based code transformations published in the literature is the lack of exploration of the effect of different data input on the reported results. An interesting question is how misleading a performance prediction that uses a single data input may be.

The goal of this section is to investigate the potential error in the prediction for the case of \FDI using combined profiling. We designed the following experiment to compare an \FDI\ with the standard inliner from \llvm: (1) Select a reasonable set of data inputs for a given benchmark; (2) Execute all combinations of single-input profiling/single-input testing for the \FDO\ inliners, repeating each test run a number of times that is sufficient to capture runtime variances;\footnote{For the experiments described in this paper an empirical statistical study using 1000 runs revealed that three runs were sufficient.}  (3) Run the \llvm\ inliner on all inputs --- the same number of times as in (2) for each input; (4) To illustrate the best performance of \FDI\ that could be reported from the data, select the best run amongst all profiling/testing combinations for a given test input and compare with the worst run for the \llvm\ inliner; (5) To demonstrate the worst performance of \FDI, do the opposite, look for the worst \FDI\ run and the best \llvm\ run for a given test input; (6) To find what the actual comparison is, use all but the test input to generate a combined profile and use this combined profile in \FDI; (7) execute this binary the necessary number of times and compare the average of these runs with the average of the same number of runs using the \llvm\ inliner.

%======== Framework, describe

This performance evaluation uses an infrastructure based on the \llvm\ development framework. This infrastructure includes a set of C++ programs and a set of scripts to control the machine-learning training, the compilation and the exaction of performance runs. This single infrastructure offers the option of performing both single-run-training/single-run-testing \FDO and  \CP-based \FDO with multiple-run performance evaluation. The number of runs used for \CP\ and for the evaluation are parameters set by the experimenter~\cite{BerubePhD}.

%======== Setting: machines, inputs, describe

The experiments were conducted on $20$ Dell Optiplex 755 running Slackware Linux 2.6.32.39 each equipped with Intel Duo Core E6750 2.66 GHz processors, 4 GB RAM, DVD-RW drive, Intel Pro/1000 Gb ethernet, Gigabyte GeForce 8600 video cards, and 250 GB SATA II drive. 

\subsection {Case Study 1: \gcc}

For the case study with the SPEC CPU 2006  \gcc, each program is evaluated using a 15-input workload. The eleven inputs distributed with SPEC CPU 2006 are augmented with  four SPEC 2000 benchmark programs used as input: \bzip, \lbm, \mcf, and \parser. These benchmark programs had to be converted to the single pre-processed file format required by the \gcc benchmark. 

\subsection {Case Study 1: \bzip\  and \gzip}

For the case studies with \bzip\ and \gzip, the code used is not the one distributed by SPEC, but rather fully-functional versions of these programs. Using these versions eliminates the unrealistically-simplified profiling situation where mutually-exclusive use cases are combined into a single program run. Consequently, these programs cannot do decompression and compression, or multiple levels of compression, within the same run.  These distinct use-cases must be covered by different inputs in the program workload.

The inputs for compression include images, ebooks in a variety of formats, movies in MP4 format, textual representation of proteins, audio books, and object files~\cite{BerubePhD}.

\REM{
The compression set contains the following inputs, with the compression level shown in parentheses:
\begin{itemize}

\item {\tt avernum (-3)}: The installer for the demo version of the game  ``Avernum: Escape from the Pit'' from Spiderweb Software.

\item {\tt cards (-4)}: A collection of greeting card layouts in the TIFF (uncompressed) image format.

\item {\tt ebooks (-5)}: A collection of ebooks, with and without images, and in a variety of formats, from Project Gutenberg\footnote{http://www.gutenberg.org}.

\item {\tt potemkin-mp4 (-6)}: The 1925 movie ``Bronenosets Potyomkin (Battleship Potemkin)'' in MP4 format, from the Internet Archive\footnote{http://archive.org/details/BattleshipPotemkin}.

\item {\tt proteins-1 (-7)}: A sample of 33 proteins from the RCSB Protein Data Bank database.  6 files for each protein, each stored in a different text-based format, provide different characteristics of the protein's structure\footnote{http://www.rcsb.org}.

\item {\tt revelation-ogg (-8)}: The audio book ``The Revelation of Saint John'' in OGG format, from Project Gutenberg\footnote{http://www.gutenberg.org/ebooks/22945}.

\item {\tt usrlib-so (-9)}: A collection of shared object (.so) files from {\tt /usr/lib/} of a 32-bit gentoo-linux machine.

\end{itemize}


The decompression set for each compressor uses the same base set of files, pre-compressed by the appropriate compressor at the default compression level.  The decompression set is composed of:
\begin{itemize}
\item {\tt auriel}: The ``Auriel's Retreat'' land-mass addition mod by lance4791 for the game ``The Elder Scrolls IV: Oblivion'' from Bethesda Softworks\footnote{http://planetelderscrolls.gamespy.com/View.php?view=\\ \hspace*{150 pt}OblivionMods.Detail\&id=5949}.

\item {\tt gcc-453}: The source-code archive of the \gcc\ compiler, version 4.5.3\footnote{http://gcc.gnu.org/gcc-4.5}.

\item {\tt lib-a}: A collection of library files (.a) from {\tt /lib/} of a gentoo-linux machine.  As per the gentoo development guide, a library will be installed in {\tt /lib} (boot critical) or {\tt /usr/lib} (general applications), but not both\footnote{http://devmanual.gentoo.org/general-concepts/filesystem/index.html}.

\item {\tt mohicans-ogv}: The 1920 movie ``Last of the Mohicans'' in OGV (ogg video) format, from the Internet Archive\footnote{http://archive.org/details/last\_of\_the\_mohicans\_1920}.

\item {\tt ocal-019}: The Open Clip Art Library archive, version 0.19. The images are primarily in vector-graphics formats\footnote{http://openclipart.org/collections}.

\item {\tt paintings-jpg}: A collection of watercolor paintings, in JPG format.

\item {\tt proteins-2}: A completely different sample of 157 proteins from the RCSB Protein Data Bank database, each in 6 different file formats.

\item {\tt sherlock-mp3}: The audio book ``The Adventures of Sherlock Holmes'' in MP3 format, from Project Gutenberg\footnote{http://www.gutenberg.org/ebooks/28733}.

\end{itemize}
}

%======== The inlining parameters

\REM{
As \FDI\ has a set of parameters that can vary and produce different running time for the programs, it was necessary to have a decision on the values for the set of parameter. These parameter values are the same through all benchmarks, and by using them the runtime results are quite similar to \llvm\ on average. That is why they were chosen, it is a very good parameter set for the purpose of this research.
}
%======= Reinforce the purpose

