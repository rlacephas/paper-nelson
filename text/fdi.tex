
The feedback-directed inlining (\FDI) ~\cite{BerubePhD} evaluated in this work is
fundamentally different than the existing static inliner in \llvm.
The static inliner inlines small calls to remove call overhead with
minimal increases in code size.  \FDI\ attempts to minimize the
dynamic number of instructions executed by the program by inlining the
most frequent calls.  While the static inliner considers call sites on
a function-by-function basis, \FDI\ considers the set of inlining
opportunities present at global scope in the current state of the
program.

% Remove the ``then'' keyword from if statements
\SetKwIF{If}{ElseIf}{Else}{if}{}{else if}{else}{endif}
\begin{algorithm}[t!p]
\begin{small}
  
The feedback-directed inlining (\FDI) ~\cite{BerubePhD} evaluated in this work is
fundamentally different than the existing static inliner in \llvm.
The static inliner inlines small calls to remove call overhead with
minimal increases in code size.  \FDI\ attempts to minimize the
dynamic number of instructions executed by the program by inlining the
most frequent calls.  While the static inliner considers call sites on
a function-by-function basis, \FDI\ considers the set of inlining
opportunities present at global scope in the current state of the
program.

% Remove the ``then'' keyword from if statements
\SetKwIF{If}{ElseIf}{Else}{if}{}{else if}{else}{endif}
\begin{algorithm}[t!p]
\begin{small}
  
The feedback-directed inlining (\FDI) ~\cite{BerubePhD} evaluated in this work is
fundamentally different than the existing static inliner in \llvm.
The static inliner inlines small calls to remove call overhead with
minimal increases in code size.  \FDI\ attempts to minimize the
dynamic number of instructions executed by the program by inlining the
most frequent calls.  While the static inliner considers call sites on
a function-by-function basis, \FDI\ considers the set of inlining
opportunities present at global scope in the current state of the
program.

% Remove the ``then'' keyword from if statements
\SetKwIF{If}{ElseIf}{Else}{if}{}{else if}{else}{endif}
\begin{algorithm}[t!p]
\begin{small}
  
The feedback-directed inlining (\FDI) ~\cite{BerubePhD} evaluated in this work is
fundamentally different than the existing static inliner in \llvm.
The static inliner inlines small calls to remove call overhead with
minimal increases in code size.  \FDI\ attempts to minimize the
dynamic number of instructions executed by the program by inlining the
most frequent calls.  While the static inliner considers call sites on
a function-by-function basis, \FDI\ considers the set of inlining
opportunities present at global scope in the current state of the
program.

% Remove the ``then'' keyword from if statements
\SetKwIF{If}{ElseIf}{Else}{if}{}{else if}{else}{endif}
\begin{algorithm}[t!p]
\begin{small}
  \input{Algorithms/fdi}
\end{small}
  \caption{\FDI\ worklist}
  \label{alg:fdiworklist}
\end{algorithm}

What \FDI\ does is to set the problem as one of constrained minimization,
the {\it Substitution Problem} \cite{Arnold00}. So, for a particular program, a subset of
all invocations, the values of maximum program size, maximum callee size,
etc, find a sequence of substitutions that can minimize the expected
program execution time. And this is done by following the order of the
candidate call sites in the candidates list. The list is ordered by the
score of each candidate.

The assumption is that this is an efficient reduction of the Knapsack
Problem to the Substitution Problem \cite{Arnold00,Scheifler1977}. The
Knapsack Problem is known to be NP-hard \cite{Garey1979}, so it's
intractable. To perform the reduction, we first assume that it is
possible to construct an invocation with any integral execution time
overhead. The second assumption is that it is possible to construct a
procedure of any integral size less than the maximum.

%\rr{Hidden text extracted from Scheifler1977}
%We define the maximum procedure size so that all
%substitutions are possible, but we constrain the maxi-
%mum program size increase to k. By construction, the
%execution time saved by any sequence of substitutions
%will be exactly the same as the resultant size increase,
%and so the maximum time savings possible is k. If there
%is a subsequence of S that sums to k, then there is a
%sequence of substitutions that will obtain the maximum
%time savings. Conversely, if there is a sequence that
%obtains the maximum time savings, then there is a
%subsequence of S that sums to k. Hence we can solve
%the Knapsack Problem by solving the Substitution
%Problem. Any tractable solution to the Substitution
%Problem is thus a tractable solution to the Knapsack
%Problem, for we can certainly perform the above con-
%struction in polynomial time.
%
%The reduction we have given depends on the fact
%that the maximum procedure size and maximum pro-
%gram size can be made arbitrarily large. If there are
%absolute bounds on these sizes, it may be possible to
%solve the Substitution Problem in polynomial time.
%However, as long as the bounds are reasonably large,
%the polynomial will be of very high degree, and so the
%problem is still essentially intractable.
%
%In view of the above, it is reasonable to take an
%engineering approach and make certain approxima-
%tions in a manner that will arrive, we hope, at a near-
%optimal solution. To begin, we need some measure of
%size and some measure of the execution time overhead
%for any given invocation. The size change resulting
%from a substitution is easy to compute, but we need a
%method for determining how the overhead of an invo-
%cation changes as other invocations are expanded. We
%make several practical approximations to fulfill these
%needs and then return to the actual problem solution at
%the end of this section)
%
%The algorithm we choose has three distinct phases.
%At each step in the first phase, every substitution that
%will result in a nonpositive basic size change is per-
%formed, independent of the time saved; the process
%repeats until no further substitutions are possible by
%this rule. The basic size change for a substitution is the
%change in the size of the procedure body containing the
%invocation. The reduction in program size possible if
%the called procedure can be removed is n o t counted in
%the basic size change, but it/s counted in determining
%the program size after the substitution. We use the
%basic size change in an attempt to keep procedure sizes
%small. As a heuristic, we would rather substitute a small
%frequently called procedure often than expand one in-
%vocation of a very large procedure within that small
%procedure.
%
%The second phase is the heart of the algorithm. At
%each step, the invocation with the highest ratio of
%expected executions to basic size change, whose expan-
%sion will not violate the procedure size constraint, is
%expanded. Any invocations created by the substitution
%are available at the next step. The order of substitution
%can be computed very cheaply by using this rule, but
%there are no mathematical guarantees that the result
%will be near-optimal. 4 The hope, of course, is that good
%results are obtainable for this specific application and
%that more expensive calculations are not required. Sub-
%stitutions continue according to this rule until the maxi-
%mum program size is just exceeded. We prefer not to
%complicate the rule to avoid violating the size con-
%straint; the maximum is, after all, only approximate,
%and in general the additional increase will be small.
%
%Once the second phase is complete, it still may be
%possible to expand some invocations without increasing
%the total program size. When a nonrecursive procedure
%has exactly one invocation to it remaining in the entire
%program and the procedure is known not to be called by
%any other mechanism, expansion of the invocation also
%allows the procedure to be removed, and little if any
%total size increase is involved. All such expansions that
%do not violate the procedure size constraint are per-
%formed in the third and final phase.

\subsection{Worklist Algorithm}

\refAlgorithm{alg:fdiworklist} presents an outline of the worklist
algorithm used by \FDI.  The algorithm uses several data structures:
\begin{description}

\item[{\it candidates:}] \hspace{15,5 pt}
  The worklist is a sorted list of candidates.
  A call site is an inlining candidate if it is a direct call, and if
  the callee does not contain a \name{setjump} nor has any previous
  attempt to inline the callee failed.  Furthermore, the call site
  must have executed at least once during profiling.

\item[{\it ignored:}] \hspace{5 pt}
  A list of call sites that are not inlining
  candidates.  This list is maintained to enable correct and efficient
  bookkeeping, and to allow any copies of these call sites created by
  inlining their caller to be immediately ignored.

\item[{\it callers:}] \hspace{2 pt}
  A mapping from functions to the call sites that
  call them.  This map allows for the re-scoring of call sites on the
  event that a call is inlined into their callee.  That inlining will
  change the callee's size, and may change the expected
  simplifications possible if the callee is inlined.

\item[{\it inlineResult:}] \hspace{19,5 pt}
  A structure returned by inliner that
  provides summary information regarding the transformation.  In
  particular, it indicates if the attempted inlining failed.
  \FDI\ enhances the default \llvm\ structure with co-indexed lists
  identifying the new call sites created in the caller by inlining,
  and their originating call sites in the callee.  This information is
  required so that profile information can be estimated for the new
  call sites.

\end{description}

At the start of \FDI, the \CProf\ is read in, and the histograms are
associated with the appropriate call sites (\refLine{wl:init}).  Every
call site is inserted into the callers list of their callee.  During
initialization, each call site is evaluated, and added to either the
candidates or ignored list, as appropriate.  When a call site is
rejected for inlining, it is immediately and permanently moved from
the list of candidates to the ignore list.  Transformations such as
constant propagation or alias analysis can resolve the callee of an
indirect call to a single possibility, thereby making it a direct
call.  However, if the call is indirect when the call site is first
discovered by the inliner, it is placed on the ignore list in spite of
the possibility of future inlining resolving the call.  Calls to
libraries and compiler built-in functions are also immediately ignored
because they cannot be inlined.

\subsection{Candidate Scoring}
\label{inlining:scoring}

The \llvm\ inliner makes inlining decisions at each call site by
comparing the expected code growth to a fixed threshold.  \FDI\ takes
a more directly execution-time-oriented approach to inlining and
attempts to achieve the greatest reduction in executed instructions
for the least amount of code growth.  Therefore, \FDI\ breaks the
evaluation of a call site into three components: the expected inlining
benefit, the expected code growth, and execution frequency of the call
site.  Given a call site, \CS, the inlining candidate scoring
function, \ScoreOf{\CS}, combines these three elements so
that \refAlgorithm{alg:fdiworklist} can select the best (highest
score) candidates for inlining first. \CP\ provides a rich
characterization of execution frequency. Making use of that
information is described in detail in \cite{BerubePhD}; for
now, let $\RewardOf{\BenefitOf{\CS}, R_{\CS}}$ represent some function
of the estimated (execution-frequency independent) inlining benefit at
call site \CS\ and $R_{\CS}$, that call-site's \CP\ monitor.  Given
the benefit function \BenefitOf{\CS} and a cost function \CostOf{\CS}
described in this section, an inlining candidate's \Score\ is
conceptually computed:

$$ \ScoreOf{\CS} = \dfrac{\RewardOf{\BenefitOf{\CS}, R_{\CS}}}{\CostOf{\CS}} $$

More details on Benefit and Cost functions can be found in \cite{BerubePhD}.

\subsection{Finding the Best Candidate}
\label{inlining:candidate}

The best candidate to inlining is found by its own score. The way it is
done is simply sorting the candidates list by their score values. So
every time a new candidate is needed, just take the top one from the
ordered list of candidates. Line $05$ of \refAlgorithm{alg:fdiworklist}
shows this choice.

But the ordering depends on scoring, and scoring is sensitive to some
parameters values. For instance, the expected reduction/expansion for each
function is directly dependent of its expected size, and the reduction/
expansion define the benefit and the cost of a function, in other words, its
own score. The parameters that can have some effect on the scoring function
are the sizes of a instruction, a call, a return, a branch, an allocation,
and a block.

Depending on the values of these parameters, the scoring function returns
different values for each function, and the ordering may be changed, which
makes the inliner take different decisions. The sensitivity of the scoring
function have a direct effect on the sorting of the candidates list, which
also have a strong impact on inline decisions.

To assure good decisions the system must have a good scoring function,
meaning that its result allows a good ordering for the inliner. But, a good
ordering depends on how accurate are the estimated scores. One way we can
calibrate the scores is to run the system on some known benchmarks. This way
the parameters can be calibrated to their best values, those who produce
the best estimates.

So there is a need to find a ``sweetspot'' on these values. The first idea was
to apply machine learning techniques to search for it through a space of
possible values. But there are also some issues, we cannot afford to have a
huge number of acquisition points because the whole system takes a long time
to process. Also, optimizing a function without knowing if it is differentiable,
or convex, is not a trivial machine learning task.

But the simple choice of a candidate may be done in other ways, for instance
considering the current budget value, the expenditure of the candidates, etc.
In sectionref{sec:choice} we propose and measure the performance of two other
different algorithms for the task of choosing the best candidate for the
inliner, and compare them with the one used in \refAlgorithm{alg:fdiworklist}.
The first is a ``more complete'' implementation of the knapsack algorithm,
and the second represents the null hypotheses, is a random choice. We defined
our algorithm empirically, based on the data we have collected.

\end{small}
  \caption{\FDI\ worklist}
  \label{alg:fdiworklist}
\end{algorithm}

What \FDI\ does is to set the problem as one of constrained minimization,
the {\it Substitution Problem} \cite{Arnold00}. So, for a particular program, a subset of
all invocations, the values of maximum program size, maximum callee size,
etc, find a sequence of substitutions that can minimize the expected
program execution time. And this is done by following the order of the
candidate call sites in the candidates list. The list is ordered by the
score of each candidate.

The assumption is that this is an efficient reduction of the Knapsack
Problem to the Substitution Problem \cite{Arnold00,Scheifler1977}. The
Knapsack Problem is known to be NP-hard \cite{Garey1979}, so it's
intractable. To perform the reduction, we first assume that it is
possible to construct an invocation with any integral execution time
overhead. The second assumption is that it is possible to construct a
procedure of any integral size less than the maximum.

%\rr{Hidden text extracted from Scheifler1977}
%We define the maximum procedure size so that all
%substitutions are possible, but we constrain the maxi-
%mum program size increase to k. By construction, the
%execution time saved by any sequence of substitutions
%will be exactly the same as the resultant size increase,
%and so the maximum time savings possible is k. If there
%is a subsequence of S that sums to k, then there is a
%sequence of substitutions that will obtain the maximum
%time savings. Conversely, if there is a sequence that
%obtains the maximum time savings, then there is a
%subsequence of S that sums to k. Hence we can solve
%the Knapsack Problem by solving the Substitution
%Problem. Any tractable solution to the Substitution
%Problem is thus a tractable solution to the Knapsack
%Problem, for we can certainly perform the above con-
%struction in polynomial time.
%
%The reduction we have given depends on the fact
%that the maximum procedure size and maximum pro-
%gram size can be made arbitrarily large. If there are
%absolute bounds on these sizes, it may be possible to
%solve the Substitution Problem in polynomial time.
%However, as long as the bounds are reasonably large,
%the polynomial will be of very high degree, and so the
%problem is still essentially intractable.
%
%In view of the above, it is reasonable to take an
%engineering approach and make certain approxima-
%tions in a manner that will arrive, we hope, at a near-
%optimal solution. To begin, we need some measure of
%size and some measure of the execution time overhead
%for any given invocation. The size change resulting
%from a substitution is easy to compute, but we need a
%method for determining how the overhead of an invo-
%cation changes as other invocations are expanded. We
%make several practical approximations to fulfill these
%needs and then return to the actual problem solution at
%the end of this section)
%
%The algorithm we choose has three distinct phases.
%At each step in the first phase, every substitution that
%will result in a nonpositive basic size change is per-
%formed, independent of the time saved; the process
%repeats until no further substitutions are possible by
%this rule. The basic size change for a substitution is the
%change in the size of the procedure body containing the
%invocation. The reduction in program size possible if
%the called procedure can be removed is n o t counted in
%the basic size change, but it/s counted in determining
%the program size after the substitution. We use the
%basic size change in an attempt to keep procedure sizes
%small. As a heuristic, we would rather substitute a small
%frequently called procedure often than expand one in-
%vocation of a very large procedure within that small
%procedure.
%
%The second phase is the heart of the algorithm. At
%each step, the invocation with the highest ratio of
%expected executions to basic size change, whose expan-
%sion will not violate the procedure size constraint, is
%expanded. Any invocations created by the substitution
%are available at the next step. The order of substitution
%can be computed very cheaply by using this rule, but
%there are no mathematical guarantees that the result
%will be near-optimal. 4 The hope, of course, is that good
%results are obtainable for this specific application and
%that more expensive calculations are not required. Sub-
%stitutions continue according to this rule until the maxi-
%mum program size is just exceeded. We prefer not to
%complicate the rule to avoid violating the size con-
%straint; the maximum is, after all, only approximate,
%and in general the additional increase will be small.
%
%Once the second phase is complete, it still may be
%possible to expand some invocations without increasing
%the total program size. When a nonrecursive procedure
%has exactly one invocation to it remaining in the entire
%program and the procedure is known not to be called by
%any other mechanism, expansion of the invocation also
%allows the procedure to be removed, and little if any
%total size increase is involved. All such expansions that
%do not violate the procedure size constraint are per-
%formed in the third and final phase.

\subsection{Worklist Algorithm}

\refAlgorithm{alg:fdiworklist} presents an outline of the worklist
algorithm used by \FDI.  The algorithm uses several data structures:
\begin{description}

\item[{\it candidates:}] \hspace{15,5 pt}
  The worklist is a sorted list of candidates.
  A call site is an inlining candidate if it is a direct call, and if
  the callee does not contain a \name{setjump} nor has any previous
  attempt to inline the callee failed.  Furthermore, the call site
  must have executed at least once during profiling.

\item[{\it ignored:}] \hspace{5 pt}
  A list of call sites that are not inlining
  candidates.  This list is maintained to enable correct and efficient
  bookkeeping, and to allow any copies of these call sites created by
  inlining their caller to be immediately ignored.

\item[{\it callers:}] \hspace{2 pt}
  A mapping from functions to the call sites that
  call them.  This map allows for the re-scoring of call sites on the
  event that a call is inlined into their callee.  That inlining will
  change the callee's size, and may change the expected
  simplifications possible if the callee is inlined.

\item[{\it inlineResult:}] \hspace{19,5 pt}
  A structure returned by inliner that
  provides summary information regarding the transformation.  In
  particular, it indicates if the attempted inlining failed.
  \FDI\ enhances the default \llvm\ structure with co-indexed lists
  identifying the new call sites created in the caller by inlining,
  and their originating call sites in the callee.  This information is
  required so that profile information can be estimated for the new
  call sites.

\end{description}

At the start of \FDI, the \CProf\ is read in, and the histograms are
associated with the appropriate call sites (\refLine{wl:init}).  Every
call site is inserted into the callers list of their callee.  During
initialization, each call site is evaluated, and added to either the
candidates or ignored list, as appropriate.  When a call site is
rejected for inlining, it is immediately and permanently moved from
the list of candidates to the ignore list.  Transformations such as
constant propagation or alias analysis can resolve the callee of an
indirect call to a single possibility, thereby making it a direct
call.  However, if the call is indirect when the call site is first
discovered by the inliner, it is placed on the ignore list in spite of
the possibility of future inlining resolving the call.  Calls to
libraries and compiler built-in functions are also immediately ignored
because they cannot be inlined.

\subsection{Candidate Scoring}
\label{inlining:scoring}

The \llvm\ inliner makes inlining decisions at each call site by
comparing the expected code growth to a fixed threshold.  \FDI\ takes
a more directly execution-time-oriented approach to inlining and
attempts to achieve the greatest reduction in executed instructions
for the least amount of code growth.  Therefore, \FDI\ breaks the
evaluation of a call site into three components: the expected inlining
benefit, the expected code growth, and execution frequency of the call
site.  Given a call site, \CS, the inlining candidate scoring
function, \ScoreOf{\CS}, combines these three elements so
that \refAlgorithm{alg:fdiworklist} can select the best (highest
score) candidates for inlining first. \CP\ provides a rich
characterization of execution frequency. Making use of that
information is described in detail in \cite{BerubePhD}; for
now, let $\RewardOf{\BenefitOf{\CS}, R_{\CS}}$ represent some function
of the estimated (execution-frequency independent) inlining benefit at
call site \CS\ and $R_{\CS}$, that call-site's \CP\ monitor.  Given
the benefit function \BenefitOf{\CS} and a cost function \CostOf{\CS}
described in this section, an inlining candidate's \Score\ is
conceptually computed:

$$ \ScoreOf{\CS} = \dfrac{\RewardOf{\BenefitOf{\CS}, R_{\CS}}}{\CostOf{\CS}} $$

More details on Benefit and Cost functions can be found in \cite{BerubePhD}.

\subsection{Finding the Best Candidate}
\label{inlining:candidate}

The best candidate to inlining is found by its own score. The way it is
done is simply sorting the candidates list by their score values. So
every time a new candidate is needed, just take the top one from the
ordered list of candidates. Line $05$ of \refAlgorithm{alg:fdiworklist}
shows this choice.

But the ordering depends on scoring, and scoring is sensitive to some
parameters values. For instance, the expected reduction/expansion for each
function is directly dependent of its expected size, and the reduction/
expansion define the benefit and the cost of a function, in other words, its
own score. The parameters that can have some effect on the scoring function
are the sizes of a instruction, a call, a return, a branch, an allocation,
and a block.

Depending on the values of these parameters, the scoring function returns
different values for each function, and the ordering may be changed, which
makes the inliner take different decisions. The sensitivity of the scoring
function have a direct effect on the sorting of the candidates list, which
also have a strong impact on inline decisions.

To assure good decisions the system must have a good scoring function,
meaning that its result allows a good ordering for the inliner. But, a good
ordering depends on how accurate are the estimated scores. One way we can
calibrate the scores is to run the system on some known benchmarks. This way
the parameters can be calibrated to their best values, those who produce
the best estimates.

So there is a need to find a ``sweetspot'' on these values. The first idea was
to apply machine learning techniques to search for it through a space of
possible values. But there are also some issues, we cannot afford to have a
huge number of acquisition points because the whole system takes a long time
to process. Also, optimizing a function without knowing if it is differentiable,
or convex, is not a trivial machine learning task.

But the simple choice of a candidate may be done in other ways, for instance
considering the current budget value, the expenditure of the candidates, etc.
In sectionref{sec:choice} we propose and measure the performance of two other
different algorithms for the task of choosing the best candidate for the
inliner, and compare them with the one used in \refAlgorithm{alg:fdiworklist}.
The first is a ``more complete'' implementation of the knapsack algorithm,
and the second represents the null hypotheses, is a random choice. We defined
our algorithm empirically, based on the data we have collected.

\end{small}
  \caption{\FDI\ worklist}
  \label{alg:fdiworklist}
\end{algorithm}

What \FDI\ does is to set the problem as one of constrained minimization,
the {\it Substitution Problem} \cite{Arnold00}. So, for a particular program, a subset of
all invocations, the values of maximum program size, maximum callee size,
etc, find a sequence of substitutions that can minimize the expected
program execution time. And this is done by following the order of the
candidate call sites in the candidates list. The list is ordered by the
score of each candidate.

The assumption is that this is an efficient reduction of the Knapsack
Problem to the Substitution Problem \cite{Arnold00,Scheifler1977}. The
Knapsack Problem is known to be NP-hard \cite{Garey1979}, so it's
intractable. To perform the reduction, we first assume that it is
possible to construct an invocation with any integral execution time
overhead. The second assumption is that it is possible to construct a
procedure of any integral size less than the maximum.

%\rr{Hidden text extracted from Scheifler1977}
%We define the maximum procedure size so that all
%substitutions are possible, but we constrain the maxi-
%mum program size increase to k. By construction, the
%execution time saved by any sequence of substitutions
%will be exactly the same as the resultant size increase,
%and so the maximum time savings possible is k. If there
%is a subsequence of S that sums to k, then there is a
%sequence of substitutions that will obtain the maximum
%time savings. Conversely, if there is a sequence that
%obtains the maximum time savings, then there is a
%subsequence of S that sums to k. Hence we can solve
%the Knapsack Problem by solving the Substitution
%Problem. Any tractable solution to the Substitution
%Problem is thus a tractable solution to the Knapsack
%Problem, for we can certainly perform the above con-
%struction in polynomial time.
%
%The reduction we have given depends on the fact
%that the maximum procedure size and maximum pro-
%gram size can be made arbitrarily large. If there are
%absolute bounds on these sizes, it may be possible to
%solve the Substitution Problem in polynomial time.
%However, as long as the bounds are reasonably large,
%the polynomial will be of very high degree, and so the
%problem is still essentially intractable.
%
%In view of the above, it is reasonable to take an
%engineering approach and make certain approxima-
%tions in a manner that will arrive, we hope, at a near-
%optimal solution. To begin, we need some measure of
%size and some measure of the execution time overhead
%for any given invocation. The size change resulting
%from a substitution is easy to compute, but we need a
%method for determining how the overhead of an invo-
%cation changes as other invocations are expanded. We
%make several practical approximations to fulfill these
%needs and then return to the actual problem solution at
%the end of this section)
%
%The algorithm we choose has three distinct phases.
%At each step in the first phase, every substitution that
%will result in a nonpositive basic size change is per-
%formed, independent of the time saved; the process
%repeats until no further substitutions are possible by
%this rule. The basic size change for a substitution is the
%change in the size of the procedure body containing the
%invocation. The reduction in program size possible if
%the called procedure can be removed is n o t counted in
%the basic size change, but it/s counted in determining
%the program size after the substitution. We use the
%basic size change in an attempt to keep procedure sizes
%small. As a heuristic, we would rather substitute a small
%frequently called procedure often than expand one in-
%vocation of a very large procedure within that small
%procedure.
%
%The second phase is the heart of the algorithm. At
%each step, the invocation with the highest ratio of
%expected executions to basic size change, whose expan-
%sion will not violate the procedure size constraint, is
%expanded. Any invocations created by the substitution
%are available at the next step. The order of substitution
%can be computed very cheaply by using this rule, but
%there are no mathematical guarantees that the result
%will be near-optimal. 4 The hope, of course, is that good
%results are obtainable for this specific application and
%that more expensive calculations are not required. Sub-
%stitutions continue according to this rule until the maxi-
%mum program size is just exceeded. We prefer not to
%complicate the rule to avoid violating the size con-
%straint; the maximum is, after all, only approximate,
%and in general the additional increase will be small.
%
%Once the second phase is complete, it still may be
%possible to expand some invocations without increasing
%the total program size. When a nonrecursive procedure
%has exactly one invocation to it remaining in the entire
%program and the procedure is known not to be called by
%any other mechanism, expansion of the invocation also
%allows the procedure to be removed, and little if any
%total size increase is involved. All such expansions that
%do not violate the procedure size constraint are per-
%formed in the third and final phase.

\subsection{Worklist Algorithm}

\refAlgorithm{alg:fdiworklist} presents an outline of the worklist
algorithm used by \FDI.  The algorithm uses several data structures:
\begin{description}

\item[{\it candidates:}] \hspace{15,5 pt}
  The worklist is a sorted list of candidates.
  A call site is an inlining candidate if it is a direct call, and if
  the callee does not contain a \name{setjump} nor has any previous
  attempt to inline the callee failed.  Furthermore, the call site
  must have executed at least once during profiling.

\item[{\it ignored:}] \hspace{5 pt}
  A list of call sites that are not inlining
  candidates.  This list is maintained to enable correct and efficient
  bookkeeping, and to allow any copies of these call sites created by
  inlining their caller to be immediately ignored.

\item[{\it callers:}] \hspace{2 pt}
  A mapping from functions to the call sites that
  call them.  This map allows for the re-scoring of call sites on the
  event that a call is inlined into their callee.  That inlining will
  change the callee's size, and may change the expected
  simplifications possible if the callee is inlined.

\item[{\it inlineResult:}] \hspace{19,5 pt}
  A structure returned by inliner that
  provides summary information regarding the transformation.  In
  particular, it indicates if the attempted inlining failed.
  \FDI\ enhances the default \llvm\ structure with co-indexed lists
  identifying the new call sites created in the caller by inlining,
  and their originating call sites in the callee.  This information is
  required so that profile information can be estimated for the new
  call sites.

\end{description}

At the start of \FDI, the \CProf\ is read in, and the histograms are
associated with the appropriate call sites (\refLine{wl:init}).  Every
call site is inserted into the callers list of their callee.  During
initialization, each call site is evaluated, and added to either the
candidates or ignored list, as appropriate.  When a call site is
rejected for inlining, it is immediately and permanently moved from
the list of candidates to the ignore list.  Transformations such as
constant propagation or alias analysis can resolve the callee of an
indirect call to a single possibility, thereby making it a direct
call.  However, if the call is indirect when the call site is first
discovered by the inliner, it is placed on the ignore list in spite of
the possibility of future inlining resolving the call.  Calls to
libraries and compiler built-in functions are also immediately ignored
because they cannot be inlined.

\subsection{Candidate Scoring}
\label{inlining:scoring}

The \llvm\ inliner makes inlining decisions at each call site by
comparing the expected code growth to a fixed threshold.  \FDI\ takes
a more directly execution-time-oriented approach to inlining and
attempts to achieve the greatest reduction in executed instructions
for the least amount of code growth.  Therefore, \FDI\ breaks the
evaluation of a call site into three components: the expected inlining
benefit, the expected code growth, and execution frequency of the call
site.  Given a call site, \CS, the inlining candidate scoring
function, \ScoreOf{\CS}, combines these three elements so
that \refAlgorithm{alg:fdiworklist} can select the best (highest
score) candidates for inlining first. \CP\ provides a rich
characterization of execution frequency. Making use of that
information is described in detail in \cite{BerubePhD}; for
now, let $\RewardOf{\BenefitOf{\CS}, R_{\CS}}$ represent some function
of the estimated (execution-frequency independent) inlining benefit at
call site \CS\ and $R_{\CS}$, that call-site's \CP\ monitor.  Given
the benefit function \BenefitOf{\CS} and a cost function \CostOf{\CS}
described in this section, an inlining candidate's \Score\ is
conceptually computed:

$$ \ScoreOf{\CS} = \dfrac{\RewardOf{\BenefitOf{\CS}, R_{\CS}}}{\CostOf{\CS}} $$

More details on Benefit and Cost functions can be found in \cite{BerubePhD}.

\subsection{Finding the Best Candidate}
\label{inlining:candidate}

The best candidate to inlining is found by its own score. The way it is
done is simply sorting the candidates list by their score values. So
every time a new candidate is needed, just take the top one from the
ordered list of candidates. Line $05$ of \refAlgorithm{alg:fdiworklist}
shows this choice.

But the ordering depends on scoring, and scoring is sensitive to some
parameters values. For instance, the expected reduction/expansion for each
function is directly dependent of its expected size, and the reduction/
expansion define the benefit and the cost of a function, in other words, its
own score. The parameters that can have some effect on the scoring function
are the sizes of a instruction, a call, a return, a branch, an allocation,
and a block.

Depending on the values of these parameters, the scoring function returns
different values for each function, and the ordering may be changed, which
makes the inliner take different decisions. The sensitivity of the scoring
function have a direct effect on the sorting of the candidates list, which
also have a strong impact on inline decisions.

To assure good decisions the system must have a good scoring function,
meaning that its result allows a good ordering for the inliner. But, a good
ordering depends on how accurate are the estimated scores. One way we can
calibrate the scores is to run the system on some known benchmarks. This way
the parameters can be calibrated to their best values, those who produce
the best estimates.

So there is a need to find a ``sweetspot'' on these values. The first idea was
to apply machine learning techniques to search for it through a space of
possible values. But there are also some issues, we cannot afford to have a
huge number of acquisition points because the whole system takes a long time
to process. Also, optimizing a function without knowing if it is differentiable,
or convex, is not a trivial machine learning task.

But the simple choice of a candidate may be done in other ways, for instance
considering the current budget value, the expenditure of the candidates, etc.
In sectionref{sec:choice} we propose and measure the performance of two other
different algorithms for the task of choosing the best candidate for the
inliner, and compare them with the one used in \refAlgorithm{alg:fdiworklist}.
The first is a ``more complete'' implementation of the knapsack algorithm,
and the second represents the null hypotheses, is a random choice. We defined
our algorithm empirically, based on the data we have collected.

\end{small}
  \caption{\FDI\ worklist}
  \label{alg:fdiworklist}
\end{algorithm}

What \FDI\ does is to set the problem as one of constrained minimization,
the {\it Substitution Problem} \cite{Arnold00}. So, for a particular program, a subset of
all invocations, the values of maximum program size, maximum callee size,
etc, find a sequence of substitutions that can minimize the expected
program execution time. And this is done by following the order of the
candidate call sites in the candidates list. The list is ordered by the
score of each candidate.

The assumption is that this is an efficient reduction of the Knapsack
Problem to the Substitution Problem \cite{Arnold00,Scheifler1977}. The
Knapsack Problem is known to be NP-hard \cite{Garey1979}, so it's
intractable. To perform the reduction, we first assume that it is
possible to construct an invocation with any integral execution time
overhead. The second assumption is that it is possible to construct a
procedure of any integral size less than the maximum.

%\rr{Hidden text extracted from Scheifler1977}
%We define the maximum procedure size so that all
%substitutions are possible, but we constrain the maxi-
%mum program size increase to k. By construction, the
%execution time saved by any sequence of substitutions
%will be exactly the same as the resultant size increase,
%and so the maximum time savings possible is k. If there
%is a subsequence of S that sums to k, then there is a
%sequence of substitutions that will obtain the maximum
%time savings. Conversely, if there is a sequence that
%obtains the maximum time savings, then there is a
%subsequence of S that sums to k. Hence we can solve
%the Knapsack Problem by solving the Substitution
%Problem. Any tractable solution to the Substitution
%Problem is thus a tractable solution to the Knapsack
%Problem, for we can certainly perform the above con-
%struction in polynomial time.
%
%The reduction we have given depends on the fact
%that the maximum procedure size and maximum pro-
%gram size can be made arbitrarily large. If there are
%absolute bounds on these sizes, it may be possible to
%solve the Substitution Problem in polynomial time.
%However, as long as the bounds are reasonably large,
%the polynomial will be of very high degree, and so the
%problem is still essentially intractable.
%
%In view of the above, it is reasonable to take an
%engineering approach and make certain approxima-
%tions in a manner that will arrive, we hope, at a near-
%optimal solution. To begin, we need some measure of
%size and some measure of the execution time overhead
%for any given invocation. The size change resulting
%from a substitution is easy to compute, but we need a
%method for determining how the overhead of an invo-
%cation changes as other invocations are expanded. We
%make several practical approximations to fulfill these
%needs and then return to the actual problem solution at
%the end of this section)
%
%The algorithm we choose has three distinct phases.
%At each step in the first phase, every substitution that
%will result in a nonpositive basic size change is per-
%formed, independent of the time saved; the process
%repeats until no further substitutions are possible by
%this rule. The basic size change for a substitution is the
%change in the size of the procedure body containing the
%invocation. The reduction in program size possible if
%the called procedure can be removed is n o t counted in
%the basic size change, but it/s counted in determining
%the program size after the substitution. We use the
%basic size change in an attempt to keep procedure sizes
%small. As a heuristic, we would rather substitute a small
%frequently called procedure often than expand one in-
%vocation of a very large procedure within that small
%procedure.
%
%The second phase is the heart of the algorithm. At
%each step, the invocation with the highest ratio of
%expected executions to basic size change, whose expan-
%sion will not violate the procedure size constraint, is
%expanded. Any invocations created by the substitution
%are available at the next step. The order of substitution
%can be computed very cheaply by using this rule, but
%there are no mathematical guarantees that the result
%will be near-optimal. 4 The hope, of course, is that good
%results are obtainable for this specific application and
%that more expensive calculations are not required. Sub-
%stitutions continue according to this rule until the maxi-
%mum program size is just exceeded. We prefer not to
%complicate the rule to avoid violating the size con-
%straint; the maximum is, after all, only approximate,
%and in general the additional increase will be small.
%
%Once the second phase is complete, it still may be
%possible to expand some invocations without increasing
%the total program size. When a nonrecursive procedure
%has exactly one invocation to it remaining in the entire
%program and the procedure is known not to be called by
%any other mechanism, expansion of the invocation also
%allows the procedure to be removed, and little if any
%total size increase is involved. All such expansions that
%do not violate the procedure size constraint are per-
%formed in the third and final phase.

\subsection{Worklist Algorithm}

\refAlgorithm{alg:fdiworklist} presents an outline of the worklist
algorithm used by \FDI.  The algorithm uses several data structures:
\begin{description}

\item[{\it candidates:}] \hspace{15,5 pt}
  The worklist is a sorted list of candidates.
  A call site is an inlining candidate if it is a direct call, and if
  the callee does not contain a \name{setjump} nor has any previous
  attempt to inline the callee failed.  Furthermore, the call site
  must have executed at least once during profiling.

\item[{\it ignored:}] \hspace{5 pt}
  A list of call sites that are not inlining
  candidates.  This list is maintained to enable correct and efficient
  bookkeeping, and to allow any copies of these call sites created by
  inlining their caller to be immediately ignored.

\item[{\it callers:}] \hspace{2 pt}
  A mapping from functions to the call sites that
  call them.  This map allows for the re-scoring of call sites on the
  event that a call is inlined into their callee.  That inlining will
  change the callee's size, and may change the expected
  simplifications possible if the callee is inlined.

\item[{\it inlineResult:}] \hspace{19,5 pt}
  A structure returned by inliner that
  provides summary information regarding the transformation.  In
  particular, it indicates if the attempted inlining failed.
  \FDI\ enhances the default \llvm\ structure with co-indexed lists
  identifying the new call sites created in the caller by inlining,
  and their originating call sites in the callee.  This information is
  required so that profile information can be estimated for the new
  call sites.

\end{description}

At the start of \FDI, the \CProf\ is read in, and the histograms are
associated with the appropriate call sites (\refLine{wl:init}).  Every
call site is inserted into the callers list of their callee.  During
initialization, each call site is evaluated, and added to either the
candidates or ignored list, as appropriate.  When a call site is
rejected for inlining, it is immediately and permanently moved from
the list of candidates to the ignore list.  Transformations such as
constant propagation or alias analysis can resolve the callee of an
indirect call to a single possibility, thereby making it a direct
call.  However, if the call is indirect when the call site is first
discovered by the inliner, it is placed on the ignore list in spite of
the possibility of future inlining resolving the call.  Calls to
libraries and compiler built-in functions are also immediately ignored
because they cannot be inlined.

\subsection{Candidate Scoring}
\label{inlining:scoring}

The \llvm\ inliner makes inlining decisions at each call site by
comparing the expected code growth to a fixed threshold.  \FDI\ takes
a more directly execution-time-oriented approach to inlining and
attempts to achieve the greatest reduction in executed instructions
for the least amount of code growth.  Therefore, \FDI\ breaks the
evaluation of a call site into three components: the expected inlining
benefit, the expected code growth, and execution frequency of the call
site.  Given a call site, \CS, the inlining candidate scoring
function, \ScoreOf{\CS}, combines these three elements so
that \refAlgorithm{alg:fdiworklist} can select the best (highest
score) candidates for inlining first. \CP\ provides a rich
characterization of execution frequency. Making use of that
information is described in detail in \cite{BerubePhD}; for
now, let $\RewardOf{\BenefitOf{\CS}, R_{\CS}}$ represent some function
of the estimated (execution-frequency independent) inlining benefit at
call site \CS\ and $R_{\CS}$, that call-site's \CP\ monitor.  Given
the benefit function \BenefitOf{\CS} and a cost function \CostOf{\CS}
described in this section, an inlining candidate's \Score\ is
conceptually computed:

$$ \ScoreOf{\CS} = \dfrac{\RewardOf{\BenefitOf{\CS}, R_{\CS}}}{\CostOf{\CS}} $$

More details on Benefit and Cost functions can be found in \cite{BerubePhD}.

\subsection{Finding the Best Candidate}
\label{inlining:candidate}

The best candidate to inlining is found by its own score. The way it is
done is simply sorting the candidates list by their score values. So
every time a new candidate is needed, just take the top one from the
ordered list of candidates. Line $05$ of \refAlgorithm{alg:fdiworklist}
shows this choice.

But the ordering depends on scoring, and scoring is sensitive to some
parameters values. For instance, the expected reduction/expansion for each
function is directly dependent of its expected size, and the reduction/
expansion define the benefit and the cost of a function, in other words, its
own score. The parameters that can have some effect on the scoring function
are the sizes of a instruction, a call, a return, a branch, an allocation,
and a block.

Depending on the values of these parameters, the scoring function returns
different values for each function, and the ordering may be changed, which
makes the inliner take different decisions. The sensitivity of the scoring
function have a direct effect on the sorting of the candidates list, which
also have a strong impact on inline decisions.

To assure good decisions the system must have a good scoring function,
meaning that its result allows a good ordering for the inliner. But, a good
ordering depends on how accurate are the estimated scores. One way we can
calibrate the scores is to run the system on some known benchmarks. This way
the parameters can be calibrated to their best values, those who produce
the best estimates.

So there is a need to find a ``sweetspot'' on these values. The first idea was
to apply machine learning techniques to search for it through a space of
possible values. But there are also some issues, we cannot afford to have a
huge number of acquisition points because the whole system takes a long time
to process. Also, optimizing a function without knowing if it is differentiable,
or convex, is not a trivial machine learning task.

But the simple choice of a candidate may be done in other ways, for instance
considering the current budget value, the expenditure of the candidates, etc.
In sectionref{sec:choice} we propose and measure the performance of two other
different algorithms for the task of choosing the best candidate for the
inliner, and compare them with the one used in \refAlgorithm{alg:fdiworklist}.
The first is a ``more complete'' implementation of the knapsack algorithm,
and the second represents the null hypotheses, is a random choice. We defined
our algorithm empirically, based on the data we have collected.
